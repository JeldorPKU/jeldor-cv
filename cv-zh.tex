%% start of file `template-zh.tex'.
%% Copyright 2006-2013 Xavier Danaux (xdanaux@gmail.com).
%
% This work may be distributed and/or modified under the
% conditions of the LaTeX Project Public License version 1.3c,
% available at http://www.latex-project.org/lppl/.


\documentclass[11pt,a4paper,sans]{moderncv}   % possible options include font size ('10pt', '11pt' and '12pt'), paper size ('a4paper', 'letterpaper', 'a5paper', 'legalpaper', 'executivepaper' and 'landscape') and font family ('sans' and 'roman')

% moderncv 主题
\moderncvstyle{classic}                        % 选项参数是 ‘casual’, ‘classic’, ‘oldstyle’ 和 ’banking’
\moderncvcolor{blue}                          % 选项参数是 ‘blue’ (默认)、‘orange’、‘green’、‘red’、‘purple’ 和 ‘grey’
%\nopagenumbers{}                             % 消除注释以取消自动页码生成功能

% 字符编码
\usepackage[utf8]{inputenc}                   % 替换你正在使用的编码
\usepackage{CJKutf8}

% 调整版心
\usepackage[scale=0.8]{geometry}
%\setlength{\hintscolumnwidth}{3cm}           % 如果你希望改变日期栏的宽度

% 个人信息
\name{蒋}{雨蒙}
%\title{简历题目 (可选项)}                     % 可选项、如不需要可删除本行
\address{海淀区海淀路52号王克桢楼1619}{100871 北京}            % 可选项、如不需要可删除本行
\phone[mobile]{+86~156~0106~4647}              % 可选项、如不需要可删除本行
\email{jeldor@pku.edu.cn}                    % 可选项、如不需要可删除本行
%\homepage{www.xialongli.com}                  % 可选项、如不需要可删除本行
%\extrainfo{附加信息 (可选项)}                 % 可选项、如不需要可删除本行
\social[github]{JeldorPKU}
\sethintscolumntowidth{2012.09 -- 2016.7}
\photo[64pt][0.4pt]{picture}                  % ‘64pt’是图片必须压缩至的高度、‘0.4pt‘是图片边框的宽度 (如不需要可调节至0pt)、’picture‘ 是图片文件的名字;可选项、如不需要可删除本行
%\quote{引言(可选项)}                          % 可选项、如不需要可删除本行

% 显示索引号;仅用于在简历中使用了引言
%\makeatletter
%\renewcommand*{\bibliographyitemlabel}{\@biblabel{\arabic{enumiv}}}
%\makeatother

% 分类索引
%\usepackage{multibib}
%\newcites{book,misc}{{Books},{Others}}
%----------------------------------------------------------------------------------
%            内容
%----------------------------------------------------------------------------------
\begin{document}
\begin{CJK}{UTF8}{gbsn}                       % 详情参阅CJK文件包
\maketitle

\vspace{-1cm}
\section{教育背景}
\cventry{2012.9 -- 2016.7}{理学学士}{北京大学心理学系}{北京}{\textit{GPA 3.50}}{}  % 第3到第6编码可留白
\cventry{2016.9 至今}{硕士研究生}{北京大学心理与认知科学学院}{北京}{}{基础心理学专业认知神经科学方向,主要领域为时间知觉。}

%\section{毕业论文}
%\cvitem{题目}{\emph{题目}}
%\cvitem{导师}{导师}
%\cvitem{说明}{\small 论文简介}

\section{个人经历}

\cventry{2012.9 -- 2014.6}{北京大学心理学系学生会}{文体部干事}{}{}{参与组织元旦晚会、体育比赛等院系文体活动。}
\cventry{2013.9 -- 2014.8}{北京大学自行车协会}{主席团成员、理事}{}{}{组建执委会组织日常工作,负责筹备暑期成果交流会及暑期远征队员选拔。同时担任2014年暑期远征短途骑行团团长。}
\cventry{2014.9 至今}{北京大学自行车协会}{网络组组长}{}{}{负责协会论坛的日常维护。2015年带头编写了全国高校山地车交流赛报名网站,并负责后续开发与维护。}
\cventry{2015.9 至今}{北京大学}{助教}{}{}{担任计算概论助教,负责C语言编程的上机课程。同时担任心理学概论助教。}

\section{获得奖励}
\cvdoubleitem{2013.7}{北京大学李彦宏奖学金}{2013.8}{北京大学军训优秀学员}
\cvdoubleitem{2014.7}{北京大学三好学生}{2016.5}{ACM 程序设计竞赛三等奖}

\section{语言技能}
\cvitemwithcomment{英语}{TOEFL 107}{能够流利地用英语与他人交流,并可以无障碍进行阅读与写作。}

\section{计算机技能}
\cvitem{C/C++}{能熟练运用 C/C++ 实现简单算法并解决批处理问题。}
\cvitem{Matlab/R}{能熟练进行数据处理与可视化。}
\cvitem{\LaTeX}{能熟练使用 \LaTeX 进行一般文档的排版工作。}
\cvitem{MS Office}{会使用 Word、 Excel、 Powerpoint完成文书工作。}
\cvitem{Web}{熟悉 LAMP/LNMP 架构的网站建设,熟悉 Bootstrap 前端框架,会使用 JavaScript (jQuery)。曾用 CodeIgniter (PHP) 和 Flask (Python) 框架编写网站。}
\cvitem{Linux}{熟悉 Linux 命令行操作,会编写简单的 Shell 脚本。}

\section{个人兴趣}
\cvdoubleitem{骑行}{\small 曾两次参与北大车协暑期远征,累计骑行10000公里以上。}{计算机编程}{\small 两次参与 ACM 并曾获三等奖。自学了 Python、R 等编程语言。}

% 来自BibTeX文件但不使用multibib包的出版物
%\renewcommand*{\bibliographyitemlabel}{\@biblabel{\arabic{enumiv}}}% BibTeX的数字标签
%\nocite{*}
%\bibliographystyle{plain}
%\bibliography{publications}                    % 'publications' 是BibTeX文件的文件名

% 来自BibTeX文件并使用multibib包的出版物
%\section{出版物}
%\nocitebook{book1,book2}
%\bibliographystylebook{plain}
%\bibliographybook{publications}               % 'publications' 是BibTeX文件的文件名
%\nocitemisc{misc1,misc2,misc3}
%\bibliographystylemisc{plain}
%\bibliographymisc{publications}               % 'publications' 是BibTeX文件的文件名

\clearpage\end{CJK}
\end{document}


%% 文件结尾 `template-zh.tex'.
